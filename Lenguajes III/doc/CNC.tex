\documentclass[spanish,12pt]{article}

\usepackage{babel,t1enc,epsfig,amssymb,amscd,mathrsfs,latexsym}
\setlength{\textwidth}{6in}

\setlength{\textheight}{9in}

\setlength{\oddsidemargin}{.25in}

\setlength{\topmargin}{.25in}

\frenchspacing

\newtheorem{theorem}{Teorema}
\begin{document}


\begin{flushleft}
UNIVERSIDAD SIM\'ON BOL\'IVAR\\INGENIERIA EN COMPUTACI\'ON\\LENGUAJES III
\end{flushleft}

\vskip 1in

\begin{center}

PRIMERA ENTREGA DE LA IMPLEMENTACI\'ON DE\end{center}


\begin{center}�NC
\end{center}

\bigskip


\bigskip


\begin{center}

\end{center}

\vskip 1.25in


\begin{flushleft} 
Federico Flaviani
\hfill  Caracas, 4 de junio de 2010\\Carnet:
99-31744
\end{flushleft}


\newpage


\begin{center}{\bf INTRODUCCI\'ON}
\end{center}

\vskip 1,5cm



El objetivo de este proyecto es el de desarrollar un compilador del lenguaje �NC 
(definido por Federico Ponte y Mar\'ia Gabriela Vald\'es) para el procesardor MIPS32.

Dicho compilador esta siendo desarrollado en Java y su analizador lexicografico y sintactico 
fue hecho con las herramientas javaflex y javacup. En este informe se encontraran solamente 
detalles del back end, por lo tanto en las siguientes paginas no encontrara informaci\'on sobre 
las dos herramientas antes mensionadas.

Todos los programas en MIPS32 resultantes de la compilaci\'on de programas en �NC usados en 
este proyecto fueron verificados con el simulador Spim.

\newpage


\begin{center}{\bf ESTADO DEL C\'ODIGO}
\end{center}

\vskip 1,5cm



Solo se esta 
generando c\'odigo para imprimir la tabla de simbolos global, ninguna otra tabla de simbolos es 
impresa por el compilador. Tampoco 
se genera c\'odigo para las inicilizaciones que son hechas en el entorno global, o sea fuera del 
bloque main, de modo que si se quiere que el programa compilado corra correctamente en MIPS, por 
ahora es necesario declarar todas las variables en el entorno global y luego inicializarlas 
dentro del main.

 Hasta el momento esta implementado la generaci\'on de codigo de todas las expreciones, exepto 
las de identificadores con acceso a registro, uniones y arreglos (la generaci\'on de codigo para 
variables de tipo basico si esta implementado). 

La generaci\'on de c\'odigo de las intrucciones 
esta implementado, salvo  
la de la instrucci\'on switch  y la instrucci\'on asignaci\'on cuando el L-Value es un 
acceso a registro, union o arreglo.

Est\'a implementado el c\'alculo de los tama\~nos de los tipos, exepto el de los arreglos cuyo 
tama\~no sea declarado con una expresi\'on distinta de constante (el c\'alculo del tama\~no de los 
arreglos que son declarados con una constante si esta implementado).

No esta implementado la generaci\'on de c\'odigo para expresiones de tipo float,

\newpage

\begin{center}{\bf TECNICAS DE IMPLEMENTACI\'ON}
\end{center}
\vskip 1,5cm

Para la implementaci\'on del c\'odigo para expresiones booleanas se usan las dos tecnicas vistas 
en clases:

\begin{enumerate}
\item c\'odigo con evaluaci\'on en registros

Este tipo de c\'odigo se genera cuando la expresi\'on booleana se encuentra al lado derecho de 
una asignaci\'on.

\item c\'odigo con salto a etiquetas predeterminadas

Este tipo de c\'odigo se genera cuando la expresi\'on booleana se encuentra dentro de la 
condici\'on de un while o un if
\end{enumerate}

Las expresiones ar\'itmeticas se compilan con la t\'ecnica de evaluaci\'on en registros, tal cual 
como fue visto en clases, salvo que los atributos sintetizados de las clases son metodos de los 
arboles abstractos y los atributos heredados son parametros que se pasan a los m\'etodos de estos 
arboles. Por ejemplo:

En clases se vio que $\\$

$SUM: AEAritm \rightarrow AEAritm AEAritm\\
AEAritm_1 .prox\_reg :=AEAritm_0 .prox\_reg\\
AEAritm_2 .prox\_reg :=AEAritm_0 .prox\_reg +1\\
AEAritm_0 .reg:=AEAritm_0 .prox\_reg mod N\\
AEAritm_0 .codigo:=AEAritm_1 .codigo +\\
salv+\\
AEAritm_2 .codigo+\\
linea(AEAritm_0 .reg:=AEAritm_1 .reg+AEAritm_2 .reg)$ $\\$

Donde como se puede ver $prox\_reg$ es un atributo heredado y $codigo,reg$ son sintetizados. 

En nuestro c\'odigo en java tenemos un m\'etodo reg y uno llavado codigo, y ambos tienen como 
parametro el entero $prox\_reg$, de modo que el codigo para la suma seria:
$\\$

$left.codigo(prox\_reg)+\\
                       P.salvar(prox\_reg+1) +\\
                       right.codigo(prox\_reg +1)+\\
                       P.linea("add "+reg+", "+reg+", "+right.reg(prox\_reg +1))+\\
                       P.restaurar(prox\_reg+1)$ $\\$

Donde la clase $P$ es donde estan todos los procedimientos est\'aticos que se usaron 
en las clases de teoria, por ejemplo salvar, restaurar, newLabel, linea, etc

Los identificadores de la tabla de simbolos global (que es la \'unica a la que se le genera c\'odigo por ahora), 
se implementa como etiquetas en la secci\'on de data, los identificadores de las demas tablas se 
implementaran como direcciones en memoria calculadas por desplazamientos del $fp$.

\newpage
\begin{center}{\bf PROGRAMAS DE PRUEBA}
\end{center}
\vskip 1,5cm

El compilador CNC.class recibe como entrada un archivo ascii con extenci\'on lang y retorna 
otro archivo ascii punto s para poder ser corrido en Spim.

Los programas de prueba que se usaron estan en la carpeta nuevasPruebas y los resultados 
de compilar dichos programas estan en la carpeta resultados. Estas pruebas estan divididas en 
tres carpetas:

\begin{enumerate}
\item Expresiones: Pruebas con expresiones aritmeticas y booleanas de las cuales para estas 
ultimas existen pruebas donde se aplican evaluaci\'on en registros y otras donde se aplican 
saltos a etiquetas predeterminadas.
\item If: Pruebas con if y else anidados con otras instrucciones.
\item While: Pruebas con while anidados con las demas instrucciones.
\end{enumerate}

\end{document}
